% ==========================================
% BAB V RENCANA SELANJUTNYA
% ==========================================
\chapter{RENCANA SELANJUTNYA}
\label{chap:rencana-selanjutnya}

Rencana penelitian selanjutnya akan dilaksanakan dengan mengacu pada metodologi yang telah ditetapkan pada Bab III, yaitu studi dokumen, wawancara, observasi proses, serta kerangka analisis Six Sigma dengan pendekatan DMAIC (Define –Measure – Analyze – Improve – Control). Penggunaan beberapa teknik pengumpulan data sekaligus dilakukan untuk memastikan triangulasi data dan meningkatkan validitas temuan, sebagaimana dianjurkan dalam penelitian kualitatif \autocite{creswell2014design,bowen2009document,sugiyono2019metode}.

\section{Gambaran Umum Rencana Penelitian Selanjutnya}

Rencana penelitian selanjutnya akan dilakukan dengan mengikuti metodologi yang telah dijelaskan pada Bab III, yaitu teknik pengumpulan data berupa studi dokumen, wawancara, dan observasi proses, serta kerangka analisis DMAIC (Define–Measure–Analyze–Improve–Control) sebagai pendekatan utama dalam evaluasi sistem. Seluruh aktivitas lanjutan yang direncanakan akan disusun agar selaras dengan urutan metodologis tersebut, sehingga proses analisis, validasi, dan pengembangan rekomendasi dapat dilakukan secara sistematis dan terukur \autocite{creswell2014design,bowen2009document,pyzdek2003sixsigma}. Setelah tahap validasi awal, penelitian akan memasuki rangkaian kegiatan yang berada pada fase Define dan Measure, yaitu penyusunan dokumen kesiapan implementasi yang mencakup evaluasi kesiapan organisasi, analisis integrasi sistem, serta rencana migrasi dari sistem eksisting menuju sistem terintegrasi. Kegiatan ini dilakukan dengan menggabungkan hasil studi dokumen internal PT KPI dan wawancara dengan pemangku kepentingan terkait, sebagaimana relevan dalam penelitian kualitatif berbasis proses organisasi \autocite{sugiyono2019metode}. Tahap ini penting karena penguatan tata kelola dan kesesuaian proses merupakan fondasi transformasi pengadaan digital sebagaimana dijelaskan dalam kajian global \autocite{accenture2022energy,deloitte2023cpo}.

Pada fase Analyze, penelitian akan melakukan analisis lanjutan terhadap akar penyebab permasalahan dan gap antar proses dengan menggabungkan temuan observasi lapangan, alur kerja aktual, dan kesenjangan antara SOP dengan praktik. Selanjutnya, penelitian memasuki fase Improve, yaitu pengujian solusi mencakup pengujian fungsional dan pengujian efektivitas proses. Pengujian fungsional dilakukan untuk memastikan setiap modul mulai dari perencanaan pengadaan, penyusunan DP3, pengelolaan kontrak, kepabeanan, hingga inventori berjalan sesuai kebutuhan fungsional yang sebelumnya diidentifikasi melalui wawancara dan studi dokumen. Sementara itu, pengujian efektivitas proses akan mengacu pada indikator Six Sigma seperti cycle time, akurasi data, visibilitas end-to-end, dan pengurangan aplikasi terfragmentasi, selaras dengan prinsip data-driven improvement dalam metode DMAIC \autocite{antony2002sixsigma,ajmera2017sixsigma}. Tahap berikutnya memasuki fase Control, di mana penelitian akan mengidentifikasi risiko implementasi serta menyiapkan mekanisme pengendalian dan monitoring. Risiko yang diantisipasi meliputi hambatan integrasi dengan SAP dan iVendor, resistensi pengguna, ketidaksesuaian proses bisnis, serta potensi gangguan migrasi data. Temuan risiko ini merujuk pada literatur yang menyatakan bahwa resistensi manusia dan kompleksitas integrasi merupakan penyebab utama kegagalan transformasi digital \autocite{bienhaus2018procurement4,bcg2022supplychain}. Untuk itu, langkah mitigasi seperti sosialisasi dan pelatihan intensif, penyelarasan proses bisnis, validasi data berlapis, serta pilot project dirancang sesuai prinsip kontrol dalam DMAIC \autocite{pyzdek2003sixsigma}.

Secara keseluruhan, seluruh aktivitas lanjutan penelitian akan mengikuti urutan metodologi yang telah ditetapkan: Define untuk memfinalisasi masalah dan kebutuhan sistem; Measure untuk mengumpulkan data kuantitatif dan kualitatif terhadap hambatan eksisting; Analyze untuk menganalisis akar penyebab dan gap proses; Improve untuk menyusun, menguji, dan memvalidasi solusi; serta Control untuk menyusun mekanisme pengendalian, indikator keberhasilan, dan rekomendasi implementasi jangka panjang. 

\section{Cost–Benefit Analysis}

Cost–Benefit Analysis dilakukan untuk menilai kelayakan implementasi Sistem pengadaan digital di PT KPI. Analisis ini mencakup identifikasi seluruh komponen biaya yang muncul selama pengembangan dan penerapan sistem, serta manfaat langsung maupun tidak langsung yang dihasilkan. Dengan memahami kedua aspek tersebut, keputusan implementasi dapat dilakukan secara lebih objektif dan terukur.

\subsection{Cost Analysis}

Cost Analysis mengidentifikasi seluruh biaya yang diperlukan dalam proses pengembangan, penerapan, serta pemeliharaan Sistem pengadaan digital. Komponen biaya utama yang terlibat meliputi: biaya pengembangan perangkat lunak, implementasi integrasi sistem, penyediaan infrastruktur TI, serta pelatihan pengguna di seluruh RU. Selain itu, diperlukan pula biaya terkait migrasi data dari aplikasi lama ke platform baru agar transisi berjalan lancar. Biaya pemeliharaan rutin juga menjadi bagian penting dalam perhitungan total biaya kepemilikan. Komponen biaya utama yang terlibat meliputi:

\begin{enumerate}
\item	Initial Development and Implementation

a. Biaya yang muncul pada fase pengembangan awal, meliputi:

b. Biaya pengembangan aplikasi dan konfigurasi solusi pengadaan digital.

c. Biaya migrasi dari sistem eksisting ke sistem baru yang terintegrasi.

\item	Integration

Biaya terkait integrasi antara sistem pengadaan digital dengan aplikasi lain yang digunakan oleh holding, seperti SAP, iVendor, dan aplikasi pendukung lainnya yang telah ada di PT KPI.

\item	Change Management

a. Biaya yang diperlukan untuk kegiatan manajemen perubahan, seperti:

b. Sosialisasi dan komunikasi internal mengenai sistem baru.

c. Pelatihan (training) bagi pengguna di setiap unit.

d. Workshop dan pendampingan implementasi.

\item	Maintenance and Support

a. Biaya pemeliharaan dan dukungan berkelanjutan, mencakup:

b. Technical support dan bantuan administratif.

c. Update sistem, bug fixing, dan perbaikan error rutin.

\end{enumerate}

\subsection{Benefit Analysis}

Benefit Analysis menilai manfaat strategis dan operasional yang diperoleh setelah implementasi Sistem pengadaan digital. Manfaat ini tidak hanya berupa penghematan biaya, tetapi juga peningkatan efisiensi dan tata kelola. Selain itu, sistem terintegrasi memungkinkan proses pengadaan berjalan lebih cepat dan akurat karena alur kerja telah terdigitalisasi dan terstandarisasi. Peningkatan visibilitas data juga membantu manajemen dalam membuat keputusan strategis. Dengan demikian manfaatnya juga memperkuat kapabilitas organisasi dalam menghadapi tantangan jangka panjang. Berikut komponennya meliputi:
\begin{enumerate}
\item	Efisiensi Waktu

a. Mengurangi waktu proses manual dan perpindahan data antar sistem.

b. Alur kerja menjadi lebih sederhana, sehingga keseluruhan proses pengadaan berjalan lebih cepat.

\item	Efisiensi Biaya

a. Sentralisasi pengadaan meningkatkan kekuatan negosiasi dengan vendor.

b. Menghilangkan biaya pemeliharaan aplikasi yang tidak diperlukan, mengingat sebelumnya terdapat 26 aplikasi terpisah di PT KPI.

\item	Peningkatan Akurasi Data dan Dokumen

a. Meminimalkan kesalahan akibat entri data manual pada berbagai sistem.

b. Meningkatkan akurasi data, sehingga mendukung pengambilan keputusan yang lebih baik.

\item	Peningkatan Visibilitas

a. Memberikan visibilitas real-time terhadap seluruh aktivitas pengadaan lintas unit.

b. Mempermudah pemantauan status proses dan kinerja pengadaan.

\item	Jaminan Kepatuhan

a. Kontrol terpusat memastikan kepatuhan terhadap SOP, peraturan, dan kebijakan internal.

b. Sistem menyediakan pemeriksaan otomatis serta notifikasi untuk mencegah pelanggaran proses pengadaan.

\end{enumerate}

Berdasarkan analisis di atas, manfaat implementasi Sistem pengadaan digital secara signifikan melebihi biaya yang diperlukan. Pengurangan duplikasi sistem, efisiensi operasional, peningkatan akurasi data, serta kepatuhan proses memberikan nilai strategis yang kuat bagi PT KPI. Selain itu, konsolidasi dari 26 aplikasi menjadi satu ekosistem terintegrasi menurunkan biaya pemeliharaan dan meningkatkan efektivitas pengelolaan pengadaan. 

\section{Identifikasi Risiko dan Strategi Mitigasi}

Implementasi Sistem pengadaan digital di PT Kilang Pertamina Internasional (PT KPI) berpotensi memberikan dampak positif yang signifikan, namun dalam proses transisinya juga terdapat sejumlah risiko yang perlu diantisipasi secara strategis. Risiko-risiko ini mencakup aspek teknis, operasional, sumber daya manusia, hingga ketergantungan terhadap pihak eksternal. Oleh karena itu, identifikasi risiko dan langkah mitigasi berikut disusun agar implementasi dapat berjalan dengan efektif, minim gangguan, serta berkelanjutan.

\begin{enumerate}
\item	Resistensi Pengguna (User Resistance)

Analisis: Pengguna internal mungkin menunjukkan penolakan terhadap penggunaan sistem baru karena ketidakterbiasaan, kekhawatiran terhadap perubahan alur kerja, atau minimnya pemahaman terhadap manfaat digitalisasi.

Mitigasi: Melakukan komunikasi efektif, melibatkan pekerja dalam proses perencanaan, menyediakan pelatihan komprehensif, serta pemberian dukungan teknis sehingga tingkat penerimaan sistem meningkat.

\item	Kompleksitas Integrasi Sistem

Analisis: Integrasi dengan sistem yang sudah ada (iVendor and SAP) dapat menjadi proses yang kompleks dan berpotensi menimbulkan kendala teknis, terutama karena perbedaan arsitektur dan standar data.

Mitigasi: Melibatkan fungsi terkait seperti EIT Solution di Holding dan melakukan perencanaan integrasi secara menyeluruh sesuai prosedur. Uji coba integrasi secara bertahap juga perlu dilakukan sebelum implementasi penuh.

\item	Risiko Kesalahan Migrasi Data

Analisis: Kesalahan migrasi data dapat menyebabkan kehilangan data penting, ketidakkonsistenan data, atau gangguan pada sistem baru.

Mitigasi: Melakukan backup data, data checking, dan dry-run test migrasi secara menyeluruh. 

\item	Keterbatasan Sumber Daya (SDM dan Keahlian Teknis)

Analisis: Keterbatasan tenaga ahli atau jumlah personel dapat memperlambat proyek dan menghambat implementasi sistem baru.

Mitigasi: Melakukan perencanaan dan alokasi sumber daya secara tepat, mempertimbangkan pengadaan tenaga eksternal (jika diperlukan), serta memberikan pelatihan teknis lanjutan kepada pegawai yang terkait.

\item	Gangguan Operasional Selama Implementasi

Analisis: Implementasi sistem baru dapat menyebabkan gangguan operasional sementara, seperti keterlambatan proses approval atau akses modul.

Mitigasi: Melakukan implementasi bertahap (pilot project), menyiapkan rencana pemulihan operasional (contingency plan), dan memberikan informasi kepada pengguna mengenai potensi gangguan sementara.

\item	Kesalahan Penggunaan oleh End User

Analisis: Kesalahan end-user dapat terjadi akibat kurangnya pemahaman terhadap modul atau fitur sistem digital.

Mitigasi: Menyediakan pelatihan intensif, manual pengguna yang mudah dipahami, serta menghadirkan helpdesk responsif agar insiden dapat ditangani dengan cepat.

\item	Perubahan Kebijakan atau Regulasi

Analisis: Perubahan kebijakan Procurement atau regulasi pemerintah dapat memengaruhi proses bisnis dan persyaratan sistem.

Mitigasi: Melakukan komunikasi aktif dengan Procurement Operation, memperbarui sistem secara berkala, dan menyusun mekanisme audit kepatuhan yang terjadwal.

\item	Ketergantungan pada Vendor atau Teknologi Tertentu

Analisis: Ketergantungan pada satu penyedia teknologi dapat meningkatkan risiko apabila vendor mengalami kegagalan teknis atau kendala operasional.

Mitigasi: Melakukan diversifikasi vendor, menyiapkan rencana keberlanjutan bisnis (business continuity plan), serta memastikan kontrak layanan mencakup klausul dukungan teknis dan jaminan layanan.

\end{enumerate}


Dengan mengidentifikasi berbagai risiko dan menyediakan langkah mitigasi yang terstruktur, implementasi Pengadaan Digital di PT KPI dapat dijalankan dengan lebih aman, terukur, dan terkendali. Pendekatan mitigasi ini tidak hanya menyoroti kesiapan teknologi, tetapi juga mencakup kesiapan organisasi, kualitas data, kompetensi pengguna, serta keselarasan proses bisnis lintas fungsi. Strategi mitigasi seperti pelatihan pengguna, validasi data berlapis, penyesuaian alur kerja, serta implementasi bertahap melalui pilot project membantu meminimalkan potensi gangguan sekaligus meningkatkan penerimaan terhadap sistem baru. Dengan demikian, penerapan Pengadaan Digital tidak hanya unggul secara teknis, tetapi juga memiliki fondasi organisasi yang kuat untuk mendukung keberlanjutan transformasi.

Secara keseluruhan, rangkaian analisis pada Bab I–V menunjukkan bahwa PT KPI menghadapi permasalahan utama berupa fragmentasi aplikasi, rendahnya visibilitas proses, ketergantungan pada input manual, serta ketidakterpaduan data yang menghambat efektivitas pengadaan. Melalui studi literatur, studi dokumen, wawancara, observasi, analisis kebutuhan, hingga perancangan solusi, penelitian ini menegaskan pentingnya sistem Pengadaan Digital terintegrasi sebagai langkah strategis untuk meningkatkan efisiensi dan tata kelola pengadaan. Penyusunan kebutuhan sistem, identifikasi masalah, pemetaan solusi, serta rancangan awal perbaikan memberikan dasar yang komprehensif bagi tahap implementasi selanjutnya. Dengan demikian, rencana penelitian selanjutnya dirancang agar implementasi solusi tidak hanya menghasilkan perbaikan teknis, tetapi juga mendukung tujuan strategis PT KPI dalam membangun pengadaan yang terintegrasi dan berkelanjutan.
